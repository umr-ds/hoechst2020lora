\abstract{
In this paper, we present an approach to facilitate long-range device-to-device communication via smartphones in crisis scenarios.
Through a custom firmware for low-cost LoRa capable micro-controller boards, called \textit{rf95modem}, common devices for end users can be enabled to use LoRa through a Bluetooth, Wi-Fi, or serial connection. 
We present two applications utilizing the flexibility provided by the proposed firmware. 
First, we introduce a novel device-to-device LoRa chat application that works a) on the two major mobile platforms Android and iOS and b) on traditional computers like notebooks using a console-based interface.
Second, we demonstrate how other infrastructure-less technology can benefit from our approach by integrating it into the DTN7 delay-tolerant networking software.
The firmware, the device-to-device chat application, the integration into DTN7, as well as the experimental evaluation code fragments are available under permissive open-source licenses.
}

\keywords{
    LoRa, 
    Disaster Communication, 
    Device-To-Device Communication,
}
