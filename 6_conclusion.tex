\section{Conclusion}
\label{sec:conclusion}

In this paper, we presented an approach to integrate modern emergency and rural communications technology into existing devices. 
In particular, we presented a novel, freely available and open source modem firmware for LoRa-enabled MCUs, called \textit{rf95modem}.
Using such a companion device with our firmware, a novel device-to-device LoRa chat application for iOS, Android, and laptop/desktop computers was created. 
An integration of LoRa into the disruption-tolerant networking software DTN7 was presented.
LoRa as a protocol for such use cases was discussed, and the device-to-device chat application, as well as the DTN7 integration were experimentally evaluated.
The evaluation showed that our approach is technically feasible and enables  low-cost, low-energy, and infrastructure-less communication.
All software implemented for this paper and the results of the experimental evaluation are released with this paper under permissive open-source licenses.

There are several areas of future work. For example, to efficiently use LoRa and its limited bandwidth in crisis scenarios, a frequency plan for users and first responders should be created.
Such a plan can be integrated into the emergency communication app, and the plan could be presented to the user.
Furthermore, while the presented energy evaluation provides a basic model, further measurements with the board-specific connection options should be conducted and evaluated in field tests.
